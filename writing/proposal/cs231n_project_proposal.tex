\documentclass[12pt]{article}

% for lettered subsections
\renewcommand\thesubsection{\alph{subsection}}
% for roman numeralled subsubsections
\renewcommand\thesubsubsection{\roman{subsubsection}}

% for margins
\usepackage[margin=0.75in]{geometry}

% for figures...
\usepackage{graphicx}

% for split environment and cases
\usepackage{amsmath} 
\usepackage{amsfonts}
\usepackage{amssymb}

% for \FloatBarrier to keep figs in sections
\usepackage[section]{placeins}

% for tables
\usepackage{tabularx, caption}

% scientific notation, units
\usepackage{siunitx}

% Python Code
\usepackage{alltt, listings, textcomp, color, verbatim}

\definecolor{gray}{rgb}{0.4,0.4,0.4}
\definecolor{dkgreen}{rgb}{0,0.4,0}
\definecolor{mylilas}{RGB}{170,55,241}
\definecolor{dkblue}{rgb}{0.0,0.0,0.4}

\lstdefinestyle{python}{
	language=Python,                        % choose the language of the code
	basicstyle=\ttfamily\footnotesize,   % the size of the fonts that are used for the code
	tabsize=4,                           % sets default tabsize to 2 spaces
	showstringspaces=false,              % show spaces adding particular underscores
	showtabs=false,                      % show tabs within strings adding particular underscores
	keywordstyle=\color{blue},           % keyword style
	identifierstyle=\color{black},       % identifier style
	emphstyle=\color{black}\bf,          % emphasis style
	commentstyle=\color{dkgreen}\slshape,% comment style
	stringstyle=\color{mylilas},         % string literal style
	aboveskip=\baselineskip,             % skip space when starting code environment
	xleftmargin=10pt, xrightmargin=10pt, % code margins
	frame=lines,                         % adds a frame around the code
	numbers=none,                        % where to put the line-numbers
	numberstyle=\tiny,                   % the size of the fonts that are used for the line-numbers
	numbersep=10pt,                      % how far the line-numbers are from the code
}
\lstnewenvironment{python}{\lstset{style=python}}{}
\newcommand{\inputpython}[1]{\lstinputlisting[style=python]{#1}}

\title{CS231n Project Proposal}  
\author{Brian Rohr\\Vishal Subbiah\\Brent Lunghino}

\begin{document}

\maketitle	

Applying convolutional neural networks to label atmospheric conditions and landcover on satellite images of the Amazon rainforest. We will train our convnet to detect the cloudiness of an image, and what types of landuse are in the image, for example primary forest, water, or agriculture. This is an important challenge because it will improve automated classification of large satellite image datasets. Classified satellite data of the rainforest can be applied to compute deforestation rates, detect illegal mining/deforestation, and differentiate between natural and human-caused deforestation.
Data will be provided by Planet for a Kaggle competition. The dataset is 150,000 images that are 256x256 pixels and 4 bands of color (RGB + near-infrared). Each image is 947 m x 947 m for a resolution of 3.7 m per pixel. The dataset covers 30 million hectares of Amazon rainforest. 

We will train a convnet and use an F-score as our loss function

\begin{equation} \label{eq:F_score}
F_{\beta}=(1 + \beta^2) \frac{pr}{\beta^2 p+r}\ \ \mathrm{where}\ \ p = \frac{tp}{tp+fp},\ \ r = \frac{tp}{tp+fn},\ \beta = 1 (F_1 score).
\end{equation}
tp - true positive \\
fp - false positive \\
fn - false negative \\
p - precision \\
r - recall\\


\end{document}